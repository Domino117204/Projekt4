\documentclass[a4paper,12pt]{article}
\usepackage[utf8]{inputenc}
\usepackage{polski}
\usepackage{amsmath, amssymb}
\usepackage{graphicx}
\usepackage{tikz}
\usepackage{float}
\usepackage{caption}
\usepackage{hyperref}
\usepackage{geometry}
\geometry{margin=2.5cm}

\title{Sprawozdanie -- Projekt 4: Algorytmy z powracaniem}
\author{Laboratoria Algorytmów i Struktur Danych}
\date{\today}

\begin{document}

\maketitle

\section{Cel projektu}
Celem projektu było zaimplementowanie programu obsługującego grafy nieskierowane w dwóch trybach:
\begin{itemize}
    \item generowania grafów Hamiltonowskich o zadanym nasyceniu (30\% i 70\%),
    \item generowania grafów nie-Hamiltonowskich o nasyceniu 50\%.
\end{itemize}
Na tych grafach wykonywano operacje:
\begin{itemize}
    \item wypisania grafu,
    \item wyszukiwania cyklu Eulera,
    \item wyszukiwania cyklu Hamiltona (algorytm z powracaniem),
    \item eksportu grafu do formatu LaTeX (TikZ).
\end{itemize}

\section{Reprezentacja grafu}
Graf został zaimplementowany jako słownik sąsiedztwa \texttt{dict[int, set[int]]}. Reprezentacja ta została wybrana ze względu na:

\begin{itemize}
    \item szybki dostęp do sąsiadów wierzchołka (czas stały),
    \item łatwe dodawanie i usuwanie krawędzi,
    \item prostotę implementacji algorytmów przeszukiwania.
\end{itemize}

\section{Generowanie grafów}
\subsection{Tryb Hamiltonowski}
Tworzony jest losowy cykl Hamiltona obejmujący wszystkie $n$ wierzchołków. Następnie dodawane są krawędzie w losowych trójkątach, aż do osiągnięcia zadanej gęstości. Zapewniono parzysty stopień każdego wierzchołka (co jest warunkiem koniecznym dla istnienia cyklu Eulera).

\subsection{Tryb Nie-Hamiltonowski}
Tworzony jest graf Hamiltonowski (50\% nasycenia), po czym ostatni wierzchołek zostaje odizolowany, eliminując możliwość istnienia cyklu Hamiltona.

\section{Operacje na grafie}
Program umożliwia użytkownikowi:
\begin{enumerate}
    \item wypisanie listy sąsiedztwa,
    \item sprawdzenie istnienia cyklu Eulera (algorytm Hierholzera),
    \item wyszukiwanie cyklu Hamiltona (algorytm z powracaniem z ograniczeniem czasowym),
    \item eksport grafu do LaTeX (TikZ).
\end{enumerate}

\section{Testy i benchmarki}
\subsection{Grafy Hamiltonowskie}
Wygenerowano grafy o liczbie wierzchołków od 50 do 200 (co 10), dla nasycenia 30\%. Zmierzono czas działania algorytmu Eulera i Hamiltona. Wyniki zapisano do \texttt{czasy\_euler.csv} i \texttt{czasy\_hamilton.csv}.

\subsection{Grafy Nie-Hamiltonowskie}
Wygenerowano grafy o $n = 20, 25, 30$. Zmierzono czas działania algorytmu Hamiltona i zapisano do \texttt{czasy\_niehamilton.csv}.

\section{Wykresy}
\begin{itemize}
    \item Wykres 1: czas działania algorytmu cyklu Eulera ($t=f(n)$),
    \item Wykres 2: czas działania algorytmu Hamiltona dla grafów Hamiltonowskich,
    \item Wykres 3: czas działania algorytmu Hamiltona dla grafów nie-Hamiltonowskich.
\end{itemize}

\section{Obserwacje}
\begin{itemize}
    \item Algorytm Eulera działa bardzo szybko i jego złożoność czasowa jest liniowa względem liczby krawędzi.
    \item Algorytm Hamiltona działa znacznie wolniej, dla dużych grafów czas działania znacząco rośnie.
    \item Dla grafów nie-Hamiltonowskich algorytm nie znajduje cyklu i zużywa pełny limit czasu.
\end{itemize}

\section{Eksport grafów do LaTeX}
Program umożliwia eksport wygenerowanego grafu do pliku \texttt{.tex} z użyciem środowiska \texttt{tikzpicture}. Grafy są rozmieszczane na okręgu o stałym promieniu dla zachowania przejrzystości.

\section{Staranność wykonania}
Kod spełnia zalecenia:
\begin{itemize}
    \item czytelne nazwy zmiennych,
    \item brak duplikacji kodu,
    \item modularność (\texttt{main.py}, \texttt{graph.py}, \texttt{export.py}, \texttt{benchmark.py}),
    \item komentarze w kluczowych miejscach.
\end{itemize}

\section{Wnioski}
Projekt realizuje operacje na grafach z uwzględnieniem algorytmów klasy NP-trudnych (cykl Hamiltona) i liniowych (cykl Eulera). Efektywność algorytmów zależy od struktury grafu oraz nasycenia. Algorytmy z powracaniem są niepraktyczne dla dużych grafów bez heurystyk ograniczających przestrzeń przeszukiwań.

\end{document}
